\chapter{Chapter 2: Faster Queries using an Inverted Index}


\section{Introduction}
In this section we are evaluating three different approaches for storing the search indexes, namely array lists, hash map and tree map.\\
Inverted hash map and tree map...\\
Ran tests to compare the results, which \\

\section{Set Up}
As part of the set up of this task, the FileHelper class – specifically the \\ parseFile(String filename) method – was updated such that from the database file, only websites that have a url, title, and at least one word of webpage content are read-in and stored in the server.
This was accomplished by an IF statements to check the assignments of the url and title fields prior to adding a new Website object to the ArrayList<Website>.
However, the meat of the changes made to this method were to how the method recognised the content of each line scanned in in order to know how to treat it.
Previously, this was accomplished by making use of the knowledge of the very specific file format, String methods, and boolean field variables.
This was all replaced by two regular expressions:

Pattern website = Pattern.compile("(https?://[A-Za-z0-9./\_]+)"); \\
Pattern webTitle = Pattern.compile("[A-Z][a-z]+[A-Za-z0-9\textbackslash s]+?"); \\

and the methods of the Matcher class. Though it doesn’t look to be that big of a change, doing so means that the two field variables are no longer needed, which means less has to be juggled when reading and making further changes to the code.

\section{Indexes and Data Structures}
\subsection{SimpleIndex}
The provided default way of storing indexes was called SimpleIndex. This solution is implemented using an ArrayList<website>, which contains all of the sites and each site than have their own ArrayList<String>  containing all the words on the sites. \\

\subsection{InvertedIndexTreeMap}
The second approach to store indexes was called InvertedIndexTreeMap. Here the relationship between site and it's words is inverted, meaning that each word knows to which sites it belongs to. The underlying data structure of the TreeMap is a Red-Black tree based NavigableMap implementation, sorted either by the natural ordering of it's keys or by a Comparator. TreeMap provides \textit{guaranteed log(n) time} performance for the operations \textbf{containsKey, get, put, remove}.\cite{oracle:treemap} TreeMap use only the amount of memory needed to hold it's items, therefore this solution is suited when it is not known how many items have to be sorted in memory and there are memory limitations. Solutions is also suited when the order in which items have been stored is important and O(log n) search time is acceptable. \cite{baeldung:HashTreeCompared} 

\subsection{InvertedIndexHashMap}
The third approach to store indexes was called InvertedIndexHashMap. Also in the HashMap the relationship between the site and it's words is inverted. To accomplish this a HashMap was used, where the words were used as Keys and  a List of websites as a Value. 
The underlying data structure of the HashMap is a Hash table based implementation. This implementation gives \textit{constant-time} performance for the basic operations such as \textbf{get} and \textbf{put}. \cite{oracle:hashmap} However this is true under assumption that there are not too many collisions. This is because this Map implementation acts as a basket hash table and when buckets get too large, they get transformed into node of  TreeNodes, similar to those in TreeMap. \cite{baeldung:HashTreeCompared} Some of the downsides of building the HashMap are that it requires more memory than it is necessary to hold it's data and when a HashMap becomes full, it gets resized and rehashed, which is costly. HashMap solutions should be chosen in cases when the approximate amount of items have to be maintained in the collection is known and the order in which items have been stored is not important. \cite{baeldung:HashTreeCompared} 


\section{Benchmark} 

In order to choose one of the implementations, namely ArrayList, TreeMap or HashMap, for the Search Engine, the benchmark test was performed to gain empirical data of the performance of each of the implementations. For the benchmark test JMH, a Java harness for building, running, and analysing nano/micro/milli/macro benchmarks, was used. {OpenJDK:jmh} The benchmark  test was carried out using 20 words (random nouns, verbs, adjectives and conjunctions), which were looked-up using the three different indexes implementations and in three differnt size databases: enwiki-tiny, enwiki-small, enwiki-medium. JMH provides information about an average Score, measured in nanoseconds per operation, see results in  table \ref{table:result}\\
During the benchmark it was assured that the test environment is as similar as possible among the different trials, meaning that all tests were performed on the same machine and no other applications running on the background.

\begin{table}[!htbp]
\caption{\textbf{Benchmark Score in ns/op for SimpleIndext, InvertedIndexHashMap and InvertedIndexTreeMap using enwiki-tiny, enwiki-small and enwiki-medium data sets on average}}
\begin{tabular}{|c|c|c|c|}
\hline
\textbf{Data sets} & \textbf{SimpleIndext} & \textbf{Inv.IndexHashMap} & \textbf{Inv.IndexTreemap} \\ \hline
\textbf{} & \textbf{avgt Score ns/op} & \textbf{avgt Score ns/op} & \textbf{avgt Score ns/op} \\ \hline
enwiki-tiny &18944.884&1052.067&1591.311 \\ \hline
enwiki-small &8819338.592&1883.776&3622.582\\ \hline
enwiki-medium &233498546.571&27451.020&30176.993 \\ \hline
\end{tabular}
\label{table:result}
\end{table}


The benchmark results shows that the SimpleIndex is significantly slower than both of the Inverted Map implementations, 233498546.571 ns/op versus 27451.020 ns/op for the InvertedIndexHashMap and 30176.993 ns/op for the InvertedIndexTreeMap using the enwiki-medium dataset, respectively. 
In order to describe the resulsts let the number of websites be m and words be n. 
The difference in performance can be explained as fallows: \\
When the SimpeIndex is looking up the search word, it looks though all the sites, which takes \textit{O(m)} time, and for each site it looks through all the words which takes \textit{O(n)} time, therefore total search time is \textit{O($m\cdot n$)}. Two other methods provides faster performance time. InvertedIndexTreeMap provides a  \textit{guaranteed} performance of \textit{O(log(n))}. InvertedIndexTreeMap provides best-case performance of constant time \textit{O(1)} and the worst-case performance since the Java8 of \textit{O(log(n))} time. Worst-case performance occurs, when hash function is not implemented correctly, values are distributed poorly in buckets and there is high hash collision.



%$2 \cdot log(n) + occ$\
\subsection{Conclusion}
There are several considerations when choosing the implementation for storing the data for the Search Engine. 

HashMap seams to be better fit than a TreeMap for Search Engine solution, because in this case the order of data is not important versus the performance looking up the websites corresponding the search word is. The HashMap can be expected to perform in constant time which is better than TreeMap's \textit{log(n)} time, and only HashMap's worst-performance is be \textit{log(n)} time. The given data sets are fixed, therefore the costly resizing and rehashing is not going to occur implementing Hashmap. HashMap performed the best on all of the given different size datasets in benchmark test. This is the reasoning for choosing HashMap implementation over the TreeMap implementation for this Search Engine project.




\section{Testing}
After the above changes were implemented, development tests were written in order determine the viability of the code and whether the changes satisfied the requirements of the task. To that end, JUnit tests were devised for each class that was updated.

\subsection{JUnit tests}

\subsubsection{FileHelper Class}

White-box tests were developed around the branching statements in the updated method, and a coverage table was produced.

\begin{table}[!h]
    \caption{Coverage table of the parseFile(String filename) method}
    \begin{tabular}{|l|p{100pt}|l|}
        \hline
        \textbf{Choice} & \textbf{Input property} & \textbf{Input data set} \\ \hline
        1 catch & incorrect file name & A \\ \hline
        1 try & file name & B \\ \hline
        2 while: zero times & empty file & B1 \\ \hline
        2 while: once & file has one line & B2 \\ \hline
        2 while: more than once & file has two lines & B3 \\ \hline
        2 while: more than once & file has at least three lines & B4 \\ \hline
        3 true & the line contains a web url & B3, B4 \\ \hline
        3 false & the line does not contain a web url & B1, B2 \\ \hline
        4 true & either the listOfWords field or the title field is null & B3, B4 \\ \hline
        4 false & both the listOfWords and the title fields are not null & B4 \\ \hline
        5 true & the url field is not null & B4 \\ \hline
        5 false & the url field is null & B3, B4 \\ \hline
        6 true & the line contains a website title & B3, B4 \\ \hline
        6 false & the line doesn't contain a website title & B2 \\ \hline
        7 true & listOfWords is null & B2, B4 \\ \hline
        7 false & listOfWords is not null & B4 \\ \hline
    \end{tabular}
\end{table}

From the coverage table an expectancy table was produced.

\begin{table}[!h]
\caption{Expectancy table of the JUnit tests}
\begin{tabular}{|l|p{85pt}|p{100pt}|p{100pt}|}
\hline
\textbf{Input data set} & \textbf{Input data} & \textbf{Expected output} & \textbf{Actual output} \\ \hline
A & "wrongfilename.txt" & Exception & FileNotFoundException \\ \hline
B1 & "data/test-file1.txt" & returns an ArrayList<website>, size() == 0 & returns an ArrayList<website>, size() == 0 \\ \hline
B2 & "data/test-file2.txt" & returns an ArrayList<website>, size() == 0 & returns an ArrayList<website>, size() == 0 \\ \hline
B3 & "data/test-file3.txt" & returns an ArrayList<website>, size() == 0 & returns an ArrayList<website>, size() == 1 \\ \hline
B4 & "data/test-file-errors.txt" & returns an ArrayList<website>, size() == 2 & returns an ArrayList<website>, size() == 2 \\ \hline
B4 & "data/test-file4.txt" & returns an ArrayList<website>, size() == 2 & returns an ArrayList<website>, size() == 2 \\ \hline
\end{tabular}
\label{FH:resuts}
\end{table}

where data/test-file1.txt is an empty file, and the rest contained the following data:

\begin{table}[!h]
\begin{tabular}{|l|l|l|l|}
\hline
\textbf{data/test-file2.txt} & \textbf{data/test-file3.txt} & \textbf{data/test-file4.txt} & \textbf{data/test-file-errors.txt} \\ \hline
word3 & http://example.com & *PAGE:http://page1.com &  word1 \\
& Title1 & Title1 & word2 \\
&  & word1 & *PAGE:http://page1.com \\
&  & word2 & Title1 \\
&  & *PAGE:http://page2.com & word1 \\
&  & Title2 & word2 \\
&  & word1 &  *PAGE:http://wrong1.com \\
&  & word3 & Title1 \\
&  &  & *PAGE:http://wrong2.com \\
&  &  & *PAGE:http://wrong3.com \\
&  &  & Titleword1 Titleword2 \\
&  &  & *PAGE:http://page2.com \\
&  &  & Title2 \\
&  &  & word1 \\
&  &  & word3
\end{tabular}
\end{table}

As you can see from the Actual Output column of \ref{FH:result}, the updated code failed test B3, highlighting a weakness in the code, and subsequently had to be debugged. Including another IF statement after the while loop resolved the issue, and following that all tests were passed.

\subsubsection{InvertedIndexHashMap Class}

\subsubsection{InvertedIndexTreeMap Class}




%How to reference surce\footnotemark. 
%\footnotetext{Oracle \url{https://docs.oracle.com/javase/8/docs/api/java/util/HashMap.html}} 

%How to reference surce\footnotemark. 
%\footnotetext{Oracle \url{https://docs.oracle.com/javase/8/docs/api/java/util/TreeMap.html}} 