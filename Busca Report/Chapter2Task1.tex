\chapter{Chapter 2: Faster Queries using an Inverted Index}


\section{Introduction}
In this section we are evaluating three different approaches... lists, hash map and tree map \\
Inverted hash map and tree map...\\
Ran tests to compare the results, which \\

\section{Set Up}
As part of the set up of this task, the FileHelper class – specifically the \\ parseFile(String filename) method – was updated such that from the database file, only websites that have a url, title, and at least one word of webpage content are read-in and stored in the server.
This was accomplished by an IF statements to check the assignments of the url and title fields prior to adding a new Website object to the ArrayList<Website>.
However, the meat of the changes made to this method were to how the method recognised the content of each line scanned in in order to know how to treat it.
Previously, this was accomplished by making use of the knowledge of the very specific file format, String methods, and boolean field variables.
This was all replaced by two regular expressions:

Pattern website = Pattern.compile("(https?://[A-Za-z0-9./\_]+)"); \\
Pattern webTitle = Pattern.compile("[A-Z][a-z]+[A-Za-z0-9\textbackslash s]+?"); \\

and the methods of the Matcher class. Though it doesn’t look to be that big of a change, doing so means that the two field variables are no longer needed, which means less has to be juggled when reading and making further changes to the code.

\section{Indexes and Data Structures}
\subsection{SimpleIndex}
bla bla bla
\subsection{InvertedHashMap}
bla bla bla \cite{oracle:hashmap}
\subsection{InvertedTreeMap}
bla bla bla

\section{Analysis} 
\subsection{Search Using Lists}
\subsection{Search Usnig InvertedHashMap}
\subsection{Search Usnig InvertedTreeMap}


\section{Testing}
After the above changes were implemented, development tests were written in order determine the viability of the code and whether the changes satisfied the requirements of the task. To that end, JUnit tests were devised for each class that was updated.

\subsection{JUnit tests}

\subsubsection{FileHelper Class}

White-box tests were developed around the branching statements in the updated method, and a coverage table was produced.

\begin{table}[!h]
    \caption{Coverage table of the parseFile(String filename) method}
    \begin{tabular}{|l|p{100pt}|l|}
        \hline
        \textbf{Choice} & \textbf{Input property} & \textbf{Input data set} \\ \hline
        1 catch & incorrect file name & A \\ \hline
        1 try & file name & B \\ \hline
        2 while: zero times & empty file & B1 \\ \hline
        2 while: once & file has one line & B2 \\ \hline
        2 while: more than once & file has two lines & B3 \\ \hline
        2 while: more than once & file has at least three lines & B4 \\ \hline
        3 true & the line contains a web url & B3, B4 \\ \hline
        3 false & the line does not contain a web url & B1, B2 \\ \hline
        4 true & either the listOfWords field or the title field is null & B3, B4 \\ \hline
        4 false & both the listOfWords and the title fields are not null & B4 \\ \hline
        5 true & the url field is not null & B4 \\ \hline
        5 false & the url field is null & B3, B4 \\ \hline
        6 true & the line contains a website title & B3, B4 \\ \hline
        6 false & the line doesn't contain a website title & B2 \\ \hline
        7 true & listOfWords is null & B2, B4 \\ \hline
        7 false & listOfWords is not null & B4 \\ \hline
    \end{tabular}
\end{table}

From the coverage table an expectancy table was produced.

\begin{table}[!h]
\caption{Expectancy table of the JUnit tests}
\begin{tabular}{|l|p{85pt}|p{100pt}|p{100pt}|}
\hline
\textbf{Input data set} & \textbf{Input data} & \textbf{Expected output} & \textbf{Actual output} \\ \hline
A & "wrongfilename.txt" & Exception & FileNotFoundException \\ \hline
B1 & "data/test-file1.txt" & returns an ArrayList<website>, size() == 0 & returns an ArrayList<website>, size() == 0 \\ \hline
B2 & "data/test-file2.txt" & returns an ArrayList<website>, size() == 0 & returns an ArrayList<website>, size() == 0 \\ \hline
B3 & "data/test-file3.txt" & returns an ArrayList<website>, size() == 0 & returns an ArrayList<website>, size() == 1 \\ \hline
B4 & "data/test-file-errors.txt" & returns an ArrayList<website>, size() == 2 & returns an ArrayList<website>, size() == 2 \\ \hline
B4 & "data/test-file4.txt" & returns an ArrayList<website>, size() == 2 & returns an ArrayList<website>, size() == 2 \\ \hline
\end{tabular}
\label{FH:resuts}
\end{table}

where data/test-file1.txt is an empty file, and the rest contained the following data:

\begin{table}[!h]
\begin{tabular}{|l|l|l|l|}
\hline
\textbf{data/test-file2.txt} & \textbf{data/test-file3.txt} & \textbf{data/test-file4.txt} & \textbf{data/test-file-errors.txt} \\ \hline
word3 & http://example.com & *PAGE:http://page1.com &  word1 \\
& Title1 & Title1 & word2 \\
&  & word1 & *PAGE:http://page1.com \\
&  & word2 & Title1 \\
&  & *PAGE:http://page2.com & word1 \\
&  & Title2 & word2 \\
&  & word1 &  *PAGE:http://wrong1.com \\
&  & word3 & Title1 \\
&  &  & *PAGE:http://wrong2.com \\
&  &  & *PAGE:http://wrong3.com \\
&  &  & Titleword1 Titleword2 \\
&  &  & *PAGE:http://page2.com \\
&  &  & Title2 \\
&  &  & word1 \\
&  &  & word3
\end{tabular}
\end{table}

As you can see from the Actual Output column of \ref{FH:result}, the updated code failed test B3, highlighting a weakness in the code, and subsequently had to be debugged. Including another IF statement after the while loop resolved the issue, and following that all tests were passed.

\subsubsection{InvertedIndexHashMap Class}

\subsubsection{InvertedIndexTreeMap Class}

\subsection{Comparison Benchmarking Results}
We compared the results of ...
We made sure that that the environmetn when runnin the different test are as much as possible similar, e.g. no other programms running on the machine during the testing, that could affect the test performance results.\\
In table \ref{table:result} the result of benchmark can be seen.\\
JMH/ avg/ ns/op
link to it


\begin{table}[!htbp]
\caption{Benchmark results in nanoseconds for three type of indexes and test files }
\begin{tabular}{|c|c|c|c|}
\hline
\textbf{Test Files} & \textbf{SimpleIndext} & \textbf{Inv.IndexHashMap} & \textbf{Inv.IndexTreemap} \\ \hline
\textbf{} & \textbf{avgt Score ns/op} & \textbf{avgt Score ns/op} & \textbf{avgt Score ns/op} \\ \hline
EnWiki Tiny &18944.884&1052.067&1591.311 \\ \hline
EnWiki Small &8819338.592&1883.776&3622.582\\ \hline
EnWiki Medium &233498546.571&27451.020&30176.993 \\ \hline
\end{tabular}
\label{table:result}
\end{table}



%How to reference surce\footnotemark. 
%\footnotetext{Oracle \url{https://docs.oracle.com/javase/8/docs/api/java/util/HashMap.html}} 

%How to reference surce\footnotemark. 
%\footnotetext{Oracle \url{https://docs.oracle.com/javase/8/docs/api/java/util/TreeMap.html}} 