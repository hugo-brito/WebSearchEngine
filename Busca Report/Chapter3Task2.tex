\chapter{Refines Queries}

%• Task: A short review on the task that we had to solve.
%• Basic Approach: A basic description on how we solved the task.
%• Technical description: Description of software architecture used in the solution.
%• Testing considerations: Description of considerations for testing the correctness
%of our solution.
%• Benchmarking/Reection: Benchmarking results for experiments that allowed us
%choose the best data structure for the search engine.

%Checklist for Mandatory Tasks
%Use the following checklist to make sure your project fulfills all mandatory tasks.
% Tasks 1–3 have correspondent Java code that solve them, in particular,
% you have implemented the inverted indexes, and benchmarked them,
% your searchengine understands complex queries,
% your searchengine is able to rank results to a query, using dierent scores.
% each class and each method is documented using Javadoc
% each class is accompanied with a unit test, in which you test the public methods
%exposed by the class (no tests need to be provided for main and “getter” methods)

Responsible group member: \textbf{Hugo Brito}

Typically, a every search engine understands complex queries. This means that, the user, when searching for a website, might type in more than one search term. Moreover, there should also be possible for the user to aggregate results from different queries into one big one using a certain keyword (in our case "OR").
This chapter describes how we enabled our search engine to do so.

\section{Task}
Overview:
\itemizer
Class: QueryHandler
/**
* getMachingWebsites answers queries of the type
* "subquery1 OR subquery2 OR subquery3 ...". A "subquery"
* has the form "word1 word2 word3 ...". A website matches
* a subquery if all the words occur on the website. A website
* matches the whole query, if it matches at least one subquery.
*
* @param line the query string
* @return the list of websites that matches the query
*/
public List<Website> getMatchingWebsites(String line) {

\subsection{Basic Approach}
There is a multitude of possibilities when providing a search term on a search engine. There is no way to predict that the term will comply with what we were expecting. So it seems relevant to implement
Text \\
\subsection{Intersected Search}


\subsection{Aggregating results}

\subsection{Providing the results}

\section{Section 3.2}
Text\\




\subsection{Subsection 3.2.1}
Text\\
